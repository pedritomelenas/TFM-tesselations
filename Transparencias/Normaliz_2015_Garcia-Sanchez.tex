\documentclass{beamer}

\usetheme{default}
\beamertemplatenavigationsymbolsempty

\usepackage{euler}
\usepackage{fontspec}
\defaultfontfeatures{Mapping=tex-text}
% use main font for base text
\usefonttheme{serif}
% font for base text
%\setmainfont{Droid Serif}
\setmainfont{Optima}
\setmonofont{Monaco}

% font for title
\setbeamerfont{title}{family=\fontspec{Yanone Kaffeesatz}}

% for other elements on title page (author, date)
\setbeamerfont{title page}{family=\fontspec{Yanone Kaffeesatz}}
\setbeamerfont{frametitle}{family=\fontspec{Yanone Kaffeesatz}, size=\huge}

\definecolor{fore}{RGB}{249,242,215}
\definecolor{back}{RGB}{51,51,51}
\definecolor{title}{RGB}{255,120,0}

\setbeamercolor{titlelike}{fg=title}
\setbeamercolor{normal text}{fg=fore,bg=back}

%\definecolor{fore}{RGB}{249,242,215}
%\definecolor{back}{RGB}{51,51,51}
%\definecolor{title}{RGB}{255,120,0}

\definecolor{back}{HTML}{000222}
\definecolor{fore}{HTML}{42AFFA}
\definecolor{title}{RGB}{0,120,120}
\definecolor{MSLightBlue}{rgb}{.31,.506,.741}

\let\OldTexttt\texttt
\renewcommand{\texttt}[1]{\OldTexttt{\textcolor{fore!50}{#1}}}

\setbeamercolor{titlelike}{fg=title}
\setbeamercolor{normal text}{fg=fore!30,bg=back}

\usepackage{listings}[2013/08/05]
%\usepackage{bera}
\definecolor{keywords}{RGB}{255,0,90}
\definecolor{comments}{RGB}{60,179,113}

\setbeamertemplate{itemize item}{\color{black!50}-}
\setbeamertemplate{itemize subitem}{\color{black!50}-}


%\usepackage[latin1]{inputenc}
\usepackage{tikz}
%\usepackage{xypic}

\definecolor{color1}{RGB}{230,230,230}
\definecolor{color2}{RGB}{100,100,100}

\newcounter{totavalue}
\newcounter{parvalue}

\def\aux{1}
\def\radius{15pt}
\def\step{0pt}

\newcommand\circcounter{%
\ifnum\inserttotalframenumber<2\relax
\else
  \setcounter{totavalue}{\inserttotalframenumber}
  \setcounter{parvalue}{\insertframenumber}
  \ifnum\inserttotalframenumber>45\relax
    \renewcommand\step{0pt}
  \fi%
  \pgfmathsetmacro{\aux}{360/\thetotavalue}
  \begin{tikzpicture}[remember picture,overlay,rotate=90+\aux]
  %\foreach \i in {0,1,...,\thetotavalue}
   % \fill[color1] 
    %  (0,0) -- (-\i*\aux:\radius) arc  (-\i*\aux:-(\i+1)*\aux+\step:\radius) -- cycle;
  \ifnum\insertframenumber>0
  \foreach \i in {1,...,\insertframenumber}
    \fill[color2] 
      (0,0) -- (-\i*\aux:\radius) arc  (-\i*\aux:-(\i+1)*\aux+\step:\radius) -- cycle;
  \fi
  %\fill[white] circle (30pt);
  \node at (0,0) {\includegraphics[width=30pt]{logougr}}; %\insertframenumber}; 
  \end{tikzpicture}%
\fi%
}


%\usepackage{mdframed}
%\newenvironment{framedblock}[1]{%
%  \begin{mdframed}[linewidth=.7,skipbelow=-1pt]%, linecolor=gray]
%    \textcolor{gray}{\large{#1}}
%  \end{mdframed}
%  \begin{mdframed}[linewidth=.7,skipabove=0pt, skipbelow=5pt]
%}{
%\end{mdframed}
%}


\usepackage{mdframed}
\newenvironment{framedblock}[1]{%
  \begin{mdframed}[linewidth=.7, skipabove=10pt, skipbelow=0pt, frametitle={#1},frametitlebackgroundcolor=black!25, frametitlebelowskip=3pt, backgroundcolor=black!70, fontcolor=white]%, linecolor=gray]
}{
\end{mdframed}
\medskip
}


\lstset{ frame=Ltb,
	framerule=0pt,
	aboveskip=0.5cm,
	framextopmargin=3pt,
	framexbottommargin=3pt,
	framexleftmargin=0.4cm,
	framesep=0pt,
	rulesep=.4pt,
	backgroundcolor=\color{black!75},
	rulesepcolor=\color{black!50},
	%
	stringstyle=\ttfamily,
	showstringspaces = false,
	basicstyle=\scriptsize\ttfamily,
	commentstyle=\color{gray45},
	keywordstyle= \color{blue!30}, %\bfseries,
	%
	%     numbers=left,
	%     numbersep=15pt,
	%     numberstyle=\tiny,
	%     numberfirstline = false,
	%     breaklines=true,
}

\lstdefinelanguage{GAP}{%
    morekeywords=[2]{and,break,continue,do,elif,else,end,fail,false,fi,for,%
        function,if,in,local,mod,not,od,or,rec,repeat,return,then,true,%
        until,while, %
        AffineSemigroup, Difference, CatenaryDegreeOfAffineSemigroup, OmegaPrimalityOfAffineSemigroup, TransposedMat, FactorizationsVectorWRTList, GeneratorsOfAffineSemigroup, ElasticityOfAffineSemigroup, OmegaPrimalityOfElementInAffineSemigroup, List, Intersection, CatenaryDegreeOfElementInNumericalSemigroup, TameDegreeOfElementInNumericalSemigroup, NumericalSemigroup},%
    moredelim=[s][\color{blue!50}]{gap}{>},%
    moredelim=[s][\color{red}]{brk}{>},%
    sensitive=true,%
    morecomment=[l]\#,%
    morestring=[b]',%
    morestring=[b]",%
    }%

\title{Systems of inequalities and equations coming from factorization properties in affine semigroups}
\author{Pedro A. Garc\'{\i}a S\'anchez}
\institute{Universidad de Granada}
\date{Osnabr\"uck -May 2015}


\DeclareMathOperator{\betti}{Betti}
\DeclareMathOperator{\prim}{Prim}
\DeclareMathOperator{\ncom}{NC}
\DeclareMathOperator{\mins}{Minimals}
%\DeclareMathOperator{\ap}{Ap}
\DeclareMathOperator{\supp}{Supp}
\DeclareMathOperator{\lcm}{lcm}

%\useinnertheme{rectangles}

%\usecolortheme{wolverine}

\begin{document}

\addtocounter{framenumber}{-1}

\addtobeamertemplate{headline}{}{\vspace*{8.8cm}\hfill\circcounter\hspace*{1cm}}

\begin{frame}
\titlepage

\end{frame}



\begin{frame}{Computational tools}

\texttt{GAP} from www.gap-system.org and the package
\begin{itemize}
\item \texttt{numericalsgps} by M. Delgado, PAGS and J. J. Morais that interacts with
\begin{itemize}
\item \texttt{4ti2Interface} by S. Gutsche
\item \texttt{4ti2gap} by PAGS and Alfredo S\'anchez-R.-Navarro
\item \texttt{NormalizInterface} by S. Gutsche, M. Horn, C. S\"oger
\item \texttt{SingularInterface} by M. Barakat, M. Horn, F. L\"ubeck, O. Motsak, M. Neunhoeffer, H. Shoenemann
\item \texttt{Singular} by M. Constantini and Willlem de Graaf
\item \texttt{Graded Modules} from the homalg project
\end{itemize}
\end{itemize}

And thus
\begin{itemize}
\item \texttt{Normaliz} home.uni-osnabrueck.de/wbruns/normaliz/
\item \texttt{Singular} www.singular.uni-kl.de 
\item \texttt{4ti2} www.4ti2.de
\end{itemize}
 


\end{frame}

\begin{frame}{Scope}
	We work with \emph{atomic monoids}, that is cancellative commutative monoids $M$ with a subset $\mathcal A$ (the \emph{atoms}) such that every element $m\in M$ is expressed as a finite sum of elements of $\mathcal A$
	\medskip
	
	In particular we will be working with finitely generated monoids (the cardinality of $\mathcal A$ is finite)
	\medskip
	
	Also factorizations are studied up to units, so we will remove them and assume that $M$ is reduced (the only unit is $0$)

	
	\begin{framedblock}{Monoids considered here}
	So $M$ or a copy of $M$ lives inside $\mathbb Z_{d_1}\times \dots \times \mathbb Z_{d_r}\times \mathbb Z^k$, with $M\cap (-M)=\{0\}$, and finitely generated
	\end{framedblock}
\end{frame}


\begin{frame}[fragile]{Factorizations}
We look for the factorizations of an element $b$ in terms of the elements in $\mathcal A$

In our setting $b$ and the elements in $\mathcal A$ are in $\mathbb N^k$ for some $k$, so we have to solve the system 
\[
Ax=b
\]
where $A$ has the elements of $\mathcal A$ as columns

Hence we can use for this \texttt{Normaliz} with the option ``inhom\_equations''
\begin{lstlisting}[language=GAP, basicstyle=\ttfamily\scriptsize]
gap> FactorizationsVectorWRTList([3,3],[[2,0],[1,1],[0,2]]);
[ [ 0, 3, 0 ], [ 1, 1, 1 ] ]
\end{lstlisting}
By Dickson's Lemma the set of factorizations are finite (\emph{FF-monoids})
\end{frame}


\begin{frame}{Lengths of factorizations}
	
	Let $M$ be a monoid and $\mathcal A=\{a_1,\ldots, a_n\}$ its set of atoms 
	
	Let $m\in M$ and $m=\lambda_1 a_1+\cdots +\lambda_n a_n$ a factorization of $m$ 
	
	We identify factorizations with elements in $\mathbb N^n$, so the above factorization corresponds with $(a_1,\ldots, a_n)$
	
	\begin{framedblock}{Length}
	The \emph{length} of a factorization $z=(z_1,\ldots, z_n)$ is \[|x|=z_1+\cdots +z_n\]
	\end{framedblock}
	
	Associated invariants
	\begin{itemize}
	\item Sets of lengths
	\item Delta sets (sets of distances)
	\item Elasticity	
	\end{itemize}
	
\end{frame}

\begin{frame}{Sets of lengths of factorizations}
	
	We have seen that the set of factorizations of an element corresponds with the set of nonnegative integer solutions of a system of the form 
	\[ Ax=b\]
	If we want to calculate all factorizations with a given length $l$, then we just add the equation
	\[ x_1+\cdots +x_n=l\]
	
	Sets of lengths are very well understood and they are almost arithmetical multiprogressions 
	
	If the sets of lengths of factorizations have finitely many elements for every $m\in M$, then we say that $M$ is a \emph{BF-monoid} 
	
\end{frame}


\begin{frame}{Delta sets}
	We know that the set of lengths of factorizations of an element $m\in M$ has finitely many elements, so we can arrange them $l_1<\cdots <l_t$
	
	\begin{framedblock}{Delta sets}
	The Delta set of $m\in M$ is
	\[
	\Delta(m)=\{l_2-l_1,\ldots , l_t-l_{t-1}\}
	\]
	
	For $M$ set 
	\[
	\Delta(M)=\bigcup_{m\in M}\Delta(m)
	\]
	\end{framedblock}
	
\end{frame}

\begin{frame}{Max and min Delta values}

The minimum of the values of the Delta sets is precisely $\gcd(\Delta(M))$

The maximum is reached in what we call a Betti element

\begin{framedblock}{Betti elements}
A \emph{Betti element} of $M= \langle a_1,\ldots, a_n\rangle$ is an element $b\in M$ such that there exists two factorizations $x,y$ of $b$ for which either $(x,y)$ or $(y,x)$ is in a minimal presentation of $M$, that is, is a minimal generator of the kernel of the monoid morphism
\[
\varphi: \mathbb N^n \to M, \varphi(x_1,\ldots, x_n)=\sum x_ia_i =Ax
\]
\end{framedblock}
The set of Betti elements is denoted by $\mathrm{Betti}(M)$

\end{frame}

\begin{frame}{Characterization of Betti elements}
Let $m\in M$ and denote by $\mathsf Z(m)$ the set of factorizations of $m$

Consider the graph $\nabla_m$ as the graph with vertices the elements of $\mathsf Z(m)$, and $xy$ is an edge if $x\cdot y\neq 0$

\begin{framedblock}{Characterization}
$m\in \mathrm{Betti}(M)$ if and only if $\nabla_m$ is not connected
\end{framedblock}
 
%\begin{center}
%$\mathrm G_{26}$
%
%\begin{tikzpicture}[y=.3cm, x=.3cm,font=\sffamily]
%\draw (5,3) -- (10,3);
%
%\draw (5,3) -- (5,8);
%
%\filldraw[fill=blue!40,draw=blue!80] (5,3) circle (3pt)    node[anchor=north] {5};
%
%\filldraw[fill=blue!40,draw=blue!80] (5,8) circle (3pt)    node[anchor=south] {7};
%
%\filldraw[fill=blue!40,draw=blue!80] (10,3) circle (3pt)    node[anchor=north] {11};
%
%\filldraw[fill=blue!40,draw=blue!80] (10,8) circle (3pt)    node[anchor=south] {13};
%
%
%\end{tikzpicture}
%\end{center}
%
%$\nabla_n$ is a (nonoriented) graph with vertices the factorizations of $n$, and there is an edge if $x\cdot y\neq 0$

\begin{center}
$\nabla_{26}$, $26\in \langle 5,7,11,13\rangle$


\begin{tikzpicture}[y=.25cm, x=.25cm,font=\sffamily]


\draw (5,3) -- (5,8);

\filldraw[fill=blue!40,draw=blue!80] (5,3) circle (3pt)    node[anchor=north] {{\tiny (1,3,0,0)}};

\filldraw[fill=blue!40,draw=blue!80] (5,8) circle (3pt)    node[anchor=south] {{\tiny (3,0,1,0)}};

\filldraw[fill=blue!40,draw=blue!80] (10,5.5) circle (3pt)    node[anchor=west] {{\tiny (0,0,0,2)}};

\end{tikzpicture}
\end{center}
\end{frame}


\begin{frame}{Possible values in the Delta sets}
Let $\lambda=(\lambda_1,\ldots,\lambda_n)\in \mathbb N$ and write $X^\lambda = x_1^{\lambda_1}\cdots x^{\lambda_n}\in \mathbb K[x_1,\ldots, x_n]$ 

For $j\in \mathbb N$, set  the binomial ideal
\[I_j =\big( X^\alpha-X^\beta \mid A\alpha=A\beta,\ \big||\alpha|-|\beta|\big|\le j\big)
\]

\begin{framedblock}{Membership (C. O'Neill)}
$j\in \Delta(M)$ if and only if $I_{j-1}\subsetneq I_j$
\end{framedblock}

So we have to solve a system of the form 
\[
\begin{matrix}
(A\mid -A) (x\mid y)^T =0\\ x_1+\cdots x_n-y_1-\cdots- y_n\le j
\end{matrix}
\]
\end{frame}

\begin{frame}{Elasticity}
Assume as before that the sets of lengths of the factorizations of $m\in M$ is $\{l_1 < \cdots < l_t\}$ 

\begin{framedblock}{Elasticity}
The elasticity of $m$ is defined as 
\[
\rho(m)=\frac{l_t}{l_1}
\]
\end{framedblock}
The elasticity of $M$ is 
\[
\rho(M)=\sup\big\{ \rho(m) \mid m\in M\big\}
\]
\end{frame}

\begin{frame}{Elasticity of a monoid}

Let, as before, $A$ be the matrix containing as columns the atoms of the semigroup, and let $G$ be a Graver basis of $Ax=0$
\begin{framedblock}{Formula for the elasticity}
The elasticity of the monoid is the maximum of $\frac{|v^+|}{|v^-|}$ where $v$ ranges in $G$, and $v^+,v^-\in \mathbb N^k$ with $v=v^+-v^-$ and $v^+\cdot v^-=0$
\end{framedblock}
This is equivalent to calculate the maximum of $|x|/|y|$ where $(x,y)$ is a minimal nonzero nonnegative integer solution of 
\[
(A\mid -A)(x\mid y)^T=0
\]
A. Philipp proved that one has to look among the circuits (which can be computed by means of determinants)
\end{frame}

\begin{frame}[fragile]{The elasticity, an example}

We compute next the elasticity of $\mathcal B(\mathbb Z_2^2)$, which is the set of nonnegative integer solutions of 
\[
\begin{matrix}
\left\{
\begin{array}{c}
x+z \equiv 0 \bmod  2\\
y+z \equiv 0 \bmod 2
\end{array}
\right.
& &
\begin{pmatrix}
1 & 0 & 1 \\
0 & 1 & 1 
\end{pmatrix}
\begin{pmatrix}
x \\�y \\ z
\end{pmatrix} =0 \in \mathbb Z_2^2
\end{matrix}
\]

 \begin{lstlisting}[language=GAP]
gap> a:=AffineSemigroup("equations", [[[1,0,1],[0,1,1]],[2,2]]);    
<Affine semigroup>
gap> GeneratorsOfAffineSemigroup(a);
[ [ 0, 0, 2 ], [ 0, 2, 0 ], [ 1, 1, 1 ], [ 2, 0, 0 ] ]
gap> ElasticityOfAffineSemigroup(a);
3/2
\end{lstlisting}
In this setting 
\[
A= 
\begin{pmatrix}
0 & 0 & 1 & 2 \\
0 & 2 & 1 & 0 \\
2 & 0 & 1 & 0 
\end{pmatrix}
\]
\end{frame}

\begin{frame}{Half-factorial monoids, distances}

\begin{framedblock}{Half-factorial monoid}
The monoid $M$ is half-factorial if for every $m\in M$ its set of lengths of factorizations has cardinality one
\end{framedblock}

Equivalently, 
\begin{itemize}
\item $\Delta(M)=\emptyset$

\item $\rho(M)=1$
\end{itemize}

\begin{framedblock}{Distance}
The distance between two factorizations $x$ and $y$ of $M$  is 
\[
\mathrm d(x,y) = \max\{ |x|, |y|\} - |x\wedge y|
\]
\end{framedblock}
where $x\wedge y = (\min(x_1,y_1),\ldots, \min(x_n,y_n))$
\end{frame}


\begin{frame}{The catenary degree of an element with an example}

$66\in S= \langle 6,9,11\rangle$, $\mathsf c(S)=4$ 

\small{ The factorizations of $66\in \langle 6,9,11\rangle$ are
\[
\mathsf Z(66)=\{ (0, 0, 6 ), ( 1, 3, 3 ), ( 2, 6, 0 ), (4, 1, 3 ), ( 5, 4, 0 ),
( 8, 2, 0),( 11, 0, 0 ) \}
\]
The distance between $(11,0,0)$ and $(0,0,6)$ is $11$.

\medskip 

\begin{overprint}
\onslide<1>

\begin{tikzpicture}[y=.3cm, x=.3cm,font=\sffamily]
\draw[very thick, brown] (0,0) -- (0,10);
\draw (0,10) to[bend right] (10,10);
\draw[very thick, brown] (10,10) -- (10,0);


\filldraw[fill=blue!40,draw=blue!80] (0,0) circle (3pt)    node[anchor=north] {$(11,0,0)$};

\filldraw[fill=blue!40,draw=blue!80] (0,10) circle (3pt)    node[anchor=south] 
{$(11,0,0)$};

\filldraw[fill=blue!40,draw=blue!80] (10,10) circle (3pt)    node[anchor=south] {$( 0,0,6 )$};

\filldraw[fill=blue!40,draw=blue!80] (10,0) circle (3pt)    node[anchor=north] {$( 0,0,6 )$};

\node [below] at (5,10) {$11$};

\end{tikzpicture}


\onslide<2>
%\[
%\xymatrix @R=1pc @C=1.5pc{ (11,0,0)\ar@{-}[r]^{3}  \ar@{=}[d] &(8,2,0)
%\ar@{=}[d] \ar@{-}[r]^{10} & (0,0,6)  \ar@{=}[d]\\
%(3,0,0) & (0,2,0)/(8,2,0) & (0,0,6) }
%\]
%

\begin{tikzpicture}[y=.3cm, x=.3cm,font=\sffamily]
\draw[very 	thick, brown] (0,0) -- (0,10);
\draw (0,10) to[bend right] (10,10)  to[bend right] (20,10);
\draw[very thick, brown] (10,10) -- (10,0);
\draw[very thick, brown] (20,10) -- (20,0);


\filldraw[fill=blue!40,draw=blue!80] (0,0) circle (3pt)    node[anchor=north] {$(3,0,0)$};

\filldraw[fill=blue!40,draw=blue!80] (0,10) circle (3pt)    node[anchor=south] 
{$( 11,0,0 )$};

\filldraw[fill=blue!40,draw=blue!80] (10,10) circle (3pt)    node[anchor=south] 
{$( 8,2,0 )$};


\filldraw[fill=blue!40,draw=blue!80] (10,0) circle (3pt)    node[anchor=north] 
{$(0,2,0) | (8,2,0) $};

\filldraw[fill=blue!40,draw=blue!80] (20,0) circle (3pt)    node[anchor=north] 
{$( 0,0,6 )$};

\filldraw[fill=blue!40,draw=blue!80] (20,10) circle (3pt)    node[anchor=south] {$( 0,0,6 )$};

\node [below] at (5,10) {$3$};

\node [below] at (15,10) {$10$};


\end{tikzpicture}

\onslide<3>
%\[
%\xymatrix @C=1pc{ (11,0,0)\ar@{-}[r]^{3}  \ar@{=}[d]& (8,2,0) \ar@{=}[d]
%\ar@{-}[r]^{3} & (5,4,0) \ar@{=}[d]\ar@{-}[r]^{9}& (0,0,6)  \ar@{=}[d]\\
%...& ./(3,0,0) & (0,2,0)/(5,4,0) &(0,0,6) }
%\]

\begin{tikzpicture}[y=.3cm, x=.3cm,font=\sffamily]
\draw[very 	thick, brown] (0,0) -- (0,10);
\draw (0,10) to[bend right] (10,10)  to[bend right] (20,10) to[bend right] (30,10);
\draw[very thick, brown] (10,10) -- (10,0);
\draw[very thick, brown] (20,10) -- (20,0);
\draw[very thick, brown] (30,10) -- (30,0);


\filldraw[fill=blue!40,draw=blue!80] (0,0) circle (3pt)    node[anchor=north] {$(3,0,0)$};

\filldraw[fill=blue!40,draw=blue!80] (0,10) circle (3pt)    node[anchor=south] 
{$( 11,0,0 )$};

\filldraw[fill=blue!40,draw=blue!80] (10,10) circle (3pt)    node[anchor=south] 
{$( 8,2,0 )$};

\filldraw[fill=blue!40,draw=blue!80] (10,0) circle (3pt)    node[anchor=north] 
{$(0,2,0) | (3,0,0) $};

\filldraw[fill=blue!40,draw=blue!80] (20,10) circle (3pt)    node[anchor=south] 
{$( 5,4,0 )$};

\filldraw[fill=blue!40,draw=blue!80] (20,0) circle (3pt)    node[anchor=north] 
{$(0,2,0) | (5,4,0) $};


\filldraw[fill=blue!40,draw=blue!80] (30,0) circle (3pt)    node[anchor=north] 
{$( 0,0,6 )$};

\filldraw[fill=blue!40,draw=blue!80] (30,10) circle (3pt)    node[anchor=south] {$( 0,0,6 )$};

\node [below] at (5,10) {$3$};

\node [below] at (15,10) {$3$};

\node [below] at (25,10) {$9$};

\end{tikzpicture}


\onslide<4>
%\[
%\xymatrix @C=.5pc{ (11,0,0)\ar@{-}[r]^{3}  \ar@{=}[d]& (8,2,0) \ar@{=}[d]
%\ar@{-}[r]^{3} & (5,4,0) \ar@{=}[d]\ar@{-}[r]^{3}& (2,6,0) \ar@{=}[d]\ar@{-}[r]^{8}& (0,0,6)  \ar@{=}[d]\\
%...& ... & .../(3,0,0) & (0,2,0)/(2,6,0) &(0,0,6) }
%\]
\begin{tikzpicture}[y=.3cm, x=.3cm,font=\sffamily, every node/.style={scale=.75},scale=.75]
\draw[very 	thick, brown] (0,0) -- (0,10);
\draw (0,10) to[bend right] (10,10)  to[bend right] (20,10) to[bend right] (30,10) to[bend right] (40,10);
\draw[very thick, brown] (10,10) -- (10,0);
\draw[very thick, brown] (20,10) -- (20,0);
\draw[very thick, brown] (30,10) -- (30,0);
\draw[very thick, brown] (40,10) -- (40,0);


\filldraw[fill=blue!40,draw=blue!80] (0,0) circle (3pt)    node[anchor=north] {$(3,0,0)$};

\filldraw[fill=blue!40,draw=blue!80] (0,10) circle (3pt)    node[anchor=south] 
{$( 11,0,0 )$};

\filldraw[fill=blue!40,draw=blue!80] (10,10) circle (3pt)    node[anchor=south] 
{$( 8,2,0 )$};

\filldraw[fill=blue!40,draw=blue!80] (10,0) circle (3pt)    node[anchor=north] 
{$(0,2,0) | (3,0,0) $};

\filldraw[fill=blue!40,draw=blue!80] (20,10) circle (3pt)    node[anchor=south] 
{$( 5,4,0 )$};

\filldraw[fill=blue!40,draw=blue!80] (20,0) circle (3pt)    node[anchor=north] 
{$(0,2,0) | (3,0,0) $};

\filldraw[fill=blue!40,draw=blue!80] (30,10) circle (3pt)    node[anchor=south] 
{$( 2,6,0 )$};

\filldraw[fill=blue!40,draw=blue!80] (30,0) circle (3pt)    node[anchor=north] 
{$(0,2,0) | (2,6,0) $};


\filldraw[fill=blue!40,draw=blue!80] (40,0) circle (3pt)    node[anchor=north] 
{$( 0,0,6 )$};

\filldraw[fill=blue!40,draw=blue!80] (40,10) circle (3pt)    node[anchor=south] {$( 0,0,6 )$};

\node [below] at (5,10) {$3$};

\node [below] at (15,10) {$3$};

\node [below] at (25,10) {$3$};

\node [below] at (35,10) {$8$};

\end{tikzpicture}


\onslide<5->
%\[
%\xymatrix @C=.25pc{ (11,0,0)\ar@{-}[r]^{3}  \ar@{=}[d]& (8,2,0) \ar@{=}[d]
%\ar@{-}[r]^{3} & (5,4,0) \ar@{=}[d]\ar@{-}[r]^{3}& (2,6,0)
%\ar@{=}[d]\ar@{-}[r]^{4}&
%(1,3,3) \ar@{=}[d]\ar@{-}[r]^{4}&(0,0,6)  \ar@{=}[d]\\
%...& ... & ... & .../(1,3,0) & (0,0,3)/(1,3,0) &(0,0,3) }
%\]

\begin{tikzpicture}[y=.3cm, x=.3cm,font=\sffamily, every node/.style={scale=.65},scale=.65]
\draw[very 	thick, brown] (0,0) -- (0,10);
\draw (0,10) to[bend right] (10,10)  to[bend right] (20,10) to[bend right] (30,10) to[bend right] (40,10) to[bend right] (50,10);
\draw[very thick, brown] (10,10) -- (10,0);
\draw[very thick, brown] (20,10) -- (20,0);
\draw[very thick, brown] (30,10) -- (30,0);
\draw[very thick, brown] (40,10) -- (40,0);
\draw[very thick, brown] (50,10) -- (50,0);


\filldraw[fill=blue!40,draw=blue!80] (0,0) circle (3pt)    node[anchor=north] {$(3,0,0)$};

\filldraw[fill=blue!40,draw=blue!80] (0,10) circle (3pt)    node[anchor=south] 
{$( 11,0,0 )$};

\filldraw[fill=blue!40,draw=blue!80] (10,10) circle (3pt)    node[anchor=south] 
{$( 8,2,0 )$};

\filldraw[fill=blue!40,draw=blue!80] (10,0) circle (3pt)    node[anchor=north] 
{$(0,2,0) | (3,0,0) $};

\filldraw[fill=blue!40,draw=blue!80] (20,10) circle (3pt)    node[anchor=south] 
{$( 5,4,0 )$};

\filldraw[fill=blue!40,draw=blue!80] (20,0) circle (3pt)    node[anchor=north] 
{$(0,2,0) | (3,0,0) $};

\filldraw[fill=blue!40,draw=blue!80] (30,10) circle (3pt)    node[anchor=south] 
{$( 2,6,0 )$};

\filldraw[fill=blue!40,draw=blue!80] (30,0) circle (3pt)    node[anchor=north] 
{$(0,2,0) | (1,3,0) $};

\filldraw[fill=blue!40,draw=blue!80] (40,10) circle (3pt)    node[anchor=south] 
{$( 1,3,3 )$};

\filldraw[fill=blue!40,draw=blue!80] (40,0) circle (3pt)    node[anchor=north] 
{$(0,0,3) | (1,3,0) $};


\filldraw[fill=blue!40,draw=blue!80] (50,0) circle (3pt)    node[anchor=north] 
{$( 0,0,3 )$};

\filldraw[fill=blue!40,draw=blue!80] (50,10) circle (3pt)    node[anchor=south] {$( 0,0,6 )$};

\node [below] at (5,10) {$3$};

\node [below] at (15,10) {$3$};

\node [below] at (25,10) {$3$};

\node [below] at (35,10) {$4$};

\node [below] at (45,10) {$4$};

\end{tikzpicture}

\end{overprint}}

\end{frame}




\begin{frame}[fragile]{The catenary degree of a monoid}

Let $M$ be an affine semigroup minimally generated by $\mathcal A$

The catenary degree of $M$ is defined as 
\[
\mathsf c(M)=\max \big\{\mathsf c(m)\mid s\in M\big\}
\]

\begin{framedblock}{Calculating the catenary degree}
$\mathsf c(M)=\max\big\{ \mathsf c(m)\mid s\in \betti(M)\big\}$
\end{framedblock}

\begin{lstlisting}[language=GAP]
gap> a:=AffineSemigroup("equations",[[[1,0,1],[0,1,1]],[2,2]]);;
gap> CatenaryDegreeOfAffineSemigroup(a); 
3
\end{lstlisting}

So for the catenary degree we need a minimal presentation

\end{frame}


\begin{frame}{Other catenary degrees}

We are joining any two factorizations $x$ and $y$ of an element $m\in M$ by a chain $z_1,\ldots, z_k$ of factorizations such that $\mathrm d(z_i,z_{i+1})\le \mathrm c(m)$, the catenary degree of $m$.

\begin{itemize}
\item If we impose that $|z_1|=\cdots =|z_k|$, then we get \emph{equal} catenary degree

\item If $|z_1|\le \cdots \le |z_k|$,  we have \emph{monotone} catenary degre

\item  $|z_i|\le \max(|x|,|y|)$ yields \emph{homogeneous} catenary degree
\end{itemize}

For the equal catenary degree of $M$ we compute the catenary degree of $M^{\textsf{eq}}=\langle (a_1,1),\ldots, (a_n,1)\rangle$

For the homogeneous catenary degree $M$ that of $M^{\textsf{hom}}=\langle (a_1,1),\ldots, (a_n,1), (0,1)\rangle$

For the monotone catenary degre we need a Graver basis of 
\[
\begin{pmatrix}
A & 0\\
1 & 1 
\end{pmatrix}
\]
\end{frame}

\begin{frame}{Tame degree with an example}
We go back to $66\in S= \langle 6,9,11\rangle$, $\mathsf t(S)=7$

\medskip

\small{ The factorizations of $66\in \langle 6,9,11\rangle$ are
\[
\mathsf Z(66)=\{ (0, 0, 6 ), ( 1, 3, 3 ), ( 2, 6, 0 ), (4, 1, 3 ), ( 5, 4, 0 ),
( 8, 2, 0),( 11, 0, 0 ) \}
\]

\begin{overprint}

\onslide<1>
\begin{center}
\begin{tikzpicture}[y=.3cm, x=.3cm,font=\sffamily]

\draw (0,0) to  (10,0);


\filldraw[fill=blue!40,draw=blue!80] (00,0) circle (3pt)    node[anchor=north] {$(8,2,0)$};

\filldraw[fill=blue!40,draw=blue!80] (10,0) circle (3pt)    node[anchor=north] {$(11,  \underline{0}, 0)$};

\node [above] at (5,0) {$3$};


\end{tikzpicture}
\end{center}


\onslide<2>
\begin{center}
\begin{tikzpicture}[y=.3cm, x=.3cm,font=\sffamily]

\draw (0,0) to  (10,0);

\draw (1,0) to  (20,0);

\filldraw[fill=blue!40,draw=blue!80] (00,0) circle (3pt)    node[anchor=north] {$(8,2,0)$};

\filldraw[fill=blue!40,draw=blue!80] (10,0) circle (3pt)    node[anchor=north] {$(11,  0, \underline{0})$};

\filldraw[fill=blue!40,draw=blue!80] (20,0) circle (3pt)    node[anchor=north] {$(4,1,3)$};

\node [above] at (5,0) {$3$};

\node [above] at (15,0) {$7$};

\end{tikzpicture}
\end{center}

So the tame degree is $7$

\end{overprint}

%\begin{overprint}
%\onslide<1> Besides, {\color{red} $9$} {\em divides} $66$ \vspace{3pt}
%\[
%\xymatrix @R=1pc @C=2pc{   \\
%(11,{\color{red} 0},0)    \\
%  }
%\]
%
%\onslide<2> and {\color{blue} $11$} also {\em divides} $66$
%\[
%\xymatrix @R=1pc @C=2pc{ {\color{red} (8,2,0)}  \\
%(11,{\color{red} 0},{\color{blue} 0}) {\color{red} \ar@{-}[u]^{3}}   \\
%   }
%\]
%
%\onslide<3>
%
%\[
%\xymatrix @R=1pc @C=2pc{ {\color{red} (8,2,0)}   \\
%(11,{\color{red}0},{\color{blue} 0}) {\color{red} \ar@{-}[u]^{3}} \\
% {\color{blue} (4,1,3) \ar@{-}[u]^{7}} }
%\]
%
%\end{overprint}
}
\end{frame}

\begin{frame}{Tame degree of the monoid}

The tame degree of an affine semigroup is the maximum of the tame degrees of its elements

\begin{framedblock}{Calculating the tame degree}
$\mathsf t(M)$ is the maximum of the $\mathsf t(m)$ with $m\in M$ having associated graph $G_m$ not complete
\end{framedblock}

Where $G_m$ is the graph with vertices the atoms $a$ such that $m-a\in M$, and $ab$ is an edge if $m-(a+b)\in M$

\begin{center}
$\mathrm G_{26}$, $26\in \langle 5,7,11,13\rangle$

\begin{tikzpicture}[y=.3cm, x=.3cm,font=\sffamily]
\draw (5,3) -- (10,3);

\draw (5,3) -- (5,8);

\filldraw[fill=blue!40,draw=blue!80] (5,3) circle (3pt)    node[anchor=north] {5};

\filldraw[fill=blue!40,draw=blue!80] (5,8) circle (3pt)    node[anchor=south] {7};

\filldraw[fill=blue!40,draw=blue!80] (10,3) circle (3pt)    node[anchor=north] {11};

\filldraw[fill=blue!40,draw=blue!80] (10,8) circle (3pt)    node[anchor=south] {13};


\end{tikzpicture}
\end{center}

\end{frame}


\begin{frame}{Tame degree of the monoid, practical info}

In the numerical semigroup case, $G_s$ not complete means that $s=w+n_j$ where $w\in S\setminus\{0\}$, $w-n_i\not\in S$ for some $n_i,n_j$ atoms of the monoid

In the affine case, Ap\'ery sets are not that easy to compute, but one can still use the following fact

\begin{framedblock}{Primitive elements and tame degree}
Let $A$ be the matrix whose columns are the atoms of the monoid

The tame degree of the monoid is achieved in an element $s$ such that there exists $v=v^+-v^-$ in a Graver basis of $Ax=0$ with $\varphi(v^+)=s$
\end{framedblock}

\medskip 

So, we can use once more \texttt{4ti2} for Graver basis computations and \texttt{Normaliz} for the factorizations of each candidate $s$
\end{frame}

\begin{frame}{Tame degree and ideals}
There is an alternative way that speeds up the process, even more when our monoid is full ($\mathrm G(M)\cap \mathbb N^k=M$)

Fixed $a\in \mathcal A$, set $\mathcal M_a=\mathrm{Minimals}_\le \mathsf Z(a+M)$ and $M_a=\{A z\mid z\in \mathcal M_a\}$

\begin{framedblock}{Tame, alternate}
Let $M$ be an affine semigroup and $a\in \mathcal A$
\[ \mathsf t(M,a)=\max\big\{ \mathsf t(m,a)\mid m\in M_a\big\}\]
\end{framedblock} 

So here we need to solve a system of the form 
\[ Ax =a+Ay\]
And if the monoid $M$ is full, then 
\[Ax\ge a\]
\end{frame}

\begin{frame}[fragile]{Some examples}
 Remember that $\mathcal B(\{g_1,\ldots,g_k\})$ with $g_1,\ldots, g_k\in G$  was the set of nonnegative integer solutions of $x_1g_1+\cdots + x_kg_k=0\in G$
 
 This is a full monoid inside $\mathbb N^k$
 
\begin{lstlisting}[language=GAP, basicstyle=\ttfamily\tiny]
gap> m:=Difference(gensZ2n(3),[[0,0,0]]);
[ [ 0, 0, 1 ], [ 0, 1, 0 ], [ 0, 1, 1 ], [ 1, 0, 0 ], [ 1, 0, 1 ], 
  [ 1, 1, 0 ], [ 1, 1, 1 ] ]
gap> a:=AffineSemigroup("equations",[TransposedMat(last),[2,2,2]]);;
gap> GeneratorsOfAffineSemigroup(a);
[ [ 0, 0, 0, 0, 0, 0, 2 ], [ 0, 0, 0, 0, 0, 2, 0 ], [ 0, 0, 0, 0, 2, 0, 0 ], 
  [ 0, 0, 0, 1, 1, 1, 1 ], [ 0, 0, 0, 2, 0, 0, 0 ], [ 0, 0, 1, 0, 1, 1, 0 ], 
  [ 0, 0, 1, 1, 0, 0, 1 ], [ 0, 0, 2, 0, 0, 0, 0 ], [ 0, 1, 0, 0, 1, 0, 1 ], 
  [ 0, 1, 0, 1, 0, 1, 0 ], [ 0, 1, 1, 0, 0, 1, 1 ], [ 0, 1, 1, 1, 1, 0, 0 ], 
  [ 0, 2, 0, 0, 0, 0, 0 ], [ 1, 0, 0, 0, 0, 1, 1 ], [ 1, 0, 0, 1, 1, 0, 0 ], 
  [ 1, 0, 1, 0, 1, 0, 1 ], [ 1, 0, 1, 1, 0, 1, 0 ], [ 1, 1, 0, 0, 1, 1, 0 ], 
  [ 1, 1, 0, 1, 0, 0, 1 ], [ 1, 1, 1, 0, 0, 0, 0 ], [ 2, 0, 0, 0, 0, 0, 0 ] ]
\end{lstlisting}

The tame degree here is 4 and takes 1048757 milliseconds; taking into account that it is full it takes 4955

For $\mathbb Z_2^4$ we get 323 columns in $A$; while for $\mathbb Z_2^5$, 20367

For $\mathbb Z_2^4$ 200GB with 16 kernels... 
\end{frame}

\begin{frame}[fragile]{Periodicity?}

Some properties can be conjectured with the computer

\begin{lstlisting}[language=GAP, basicstyle=\ttfamily\tiny]
gap> s:=NumericalSemigroup(5,17,19,21);;
gap> l:=List(Intersection([1..200],s));;
gap> List(l,x->CatenaryDegreeOfElementInNumericalSemigroup(x,s));
[ 0, 0, 0, 0, 0, 0, 0, 0, 0, 0, 0, 0, 0, 0, 0, 0, 4, 0, 4, 0, 2, 4, 8, 4, 6, 
  2, 4, 8, 4, 6, 2, 4, 8, 4, 6, 4, 4, 8, 4, 6, 4, 6, 8, 5, 6, 5, 6, 8, 5, 6, 
  4, 6, 8, 5, 6, 4, 6, 8, 5, 6, 4, 6, 8, 5, 6, 4, 6, 8, 5, 6, 4, 6, 8, 5, 6, 
  4, 6, 8, 5, 6, 4, 6, 8, 5, 6, 5, 6, 8, 5, 6, 4, 6, 8, 5, 6, 4, 6, 8, 5, 6, 
  4, 6, 8, 5, 6, 4, 6, 8, 5, 6, 4, 6, 8, 5, 6, 4, 6, 8, 5, 6, 4, 6, 8, 5, 6, 
  5, 6, 8, 5, 6, 4, 6, 8, 5, 6, 4, 6, 8, 5, 6, 4, 6, 8, 5, 6, 4, 6, 8, 5, 6, 
  5, 6, 8, 5, 6, 4, 6, 8, 5, 6, 4, 6, 8, 5, 6, 4, 6, 8, 5, 6, 4, 6, 8, 5, 6, 
  4, 6, 8, 5, 6, 4, 6, 8 ]
gap> List(l,x->TameDegreeOfElementInNumericalSemigroup(x,s));          
[ 0, 0, 0, 0, 0, 0, 0, 0, 0, 0, 0, 0, 0, 0, 0, 0, 4, 0, 4, 0, 2, 4, 8, 4, 6, 
  2, 4, 8, 4, 6, 2, 4, 8, 7, 6, 4, 4, 11, 7, 8, 4, 8, 11, 5, 8, 5, 8, 11, 5, 
  8, 5, 8, 11, 5, 8, 5, 8, 11, 5, 8, 5, 8, 11, 5, 8, 4, 8, 11, 5, 8, 4, 8, 
  11, 5, 8, 4, 8, 11, 5, 8, 4, 8, 11, 5, 8, 5, 8, 11, 5, 8, 4, 8, 11, 5, 8, 
  4, 8, 11, 5, 8, 5, 8, 11, 5, 8, 4, 8, 11, 5, 8, 4, 8, 11, 5, 8, 5, 8, 11, 
  5, 8, 4, 8, 11, 5, 8, 5, 8, 11, 5, 8, 4, 8, 11, 5, 8, 4, 8, 11, 5, 8, 5, 8, 
  11, 5, 8, 4, 8, 11, 5, 8, 5, 8, 11, 5, 8, 5, 8, 11, 5, 8, 5, 8, 11, 5, 8, 
  5, 8, 11, 5, 8, 4, 8, 11, 5, 8, 4, 8, 11, 5, 8, 5, 8, 11 ]  
\end{lstlisting}

\end{frame}


\begin{frame}{The $\omega$-primality}

Let $M$ be an affine semigroup with atoms $\mathcal A=\{a_1,\ldots,a_k\}$, and let $m\in M$

The $\omega$-primality of $m$, $\omega(m)$, is the least integer $N$ such that whenever $(\sum_{i=1}^k \lambda_i a_i)-m\in M$, there exists $(\beta_1,\ldots, \beta_k)\le (\lambda_1,\ldots, \lambda_k)$ such that $(\sum_{i=1}^k \beta_i a_i)-m\in M$ and $\sum \beta_i \le N$

\begin{framedblock}{Calculating $\omega$-primality}
$\omega(M)$ is the maximum of the lengths of the minimal elements of $\mathsf Z(m+M)$
\end{framedblock}
\end{frame}

\begin{frame}[fragile]{The $\omega$-primality, practical information}

We want to calculate $\omega(s)$, and the atoms are $\mathcal A$; $A$ is a matrix with columns the elements of $\mathcal A$

\begin{itemize}
\item For numerical semigroups, one only has to look at factorizations of elements of the form $w+a$ with $w\in S\setminus\{0\}$, $w-s\not \in S$ and $a\in \mathcal A$

\item For affine semigroups, we can compute the minimals of $\mathsf Z(s+S)$ or find the solutions to  $Ax=s+Ay$,  project on $x$, and take the minimal ones
\end{itemize}

So we can  use \texttt{Normaliz}+inhom\_equations

{ 
\begin{lstlisting}[language=GAP]
gap> OmegaPrimalityOfElementInAffineSemigroup(
[1000],AffineSemigroup([[31],[51],[75],[49]]));
37
\end{lstlisting}
}

\end{frame}

\begin{frame}{Thank you!}


Thank you for producing and working with \texttt{Normaliz}

\bigskip

\noindent\textbf{Bibliography:}
\begin{itemize}
\item  A. Geroldinger and F. Halter-Koch,  Non-Unique Factorizations: Algebraic, Combinatorial, and Analytic Theory, Chapman and Hall/CRC, Boca Raton, Florida, 2006.
\item PAGS, An overview of the computational aspects of nonunique factorization invariants, arXiv:1504.07424
\end{itemize}
\end{frame}

\end{document}


\begin{frame}{Atoms in a block monoid}
	$G\cong \mathbb Z_{d_1}\times \cdots \times \mathbb Z_{d_r}$
	
	Let $g_1,\ldots, g_n\in G$. The set of zerosum sequences corresponds to the set of nonnegative integer solutions of 
	
	\[\left\{
	\begin{matrix}
	g_{11}x_1+\cdots+ g_{n_1}x_n\equiv 0 \bmod d_1\\
	\cdots \\
	g_{1r}x_1+\cdots+ g_{n_r}x_n\equiv 0 \bmod d_r
	\end{matrix}
	\right.
	\]
	
	So we can use \texttt{Normaliz} with the option ``congruences"
	
\end{frame}

\begin{frame}[fragile]{Atoms in a block monoid, example}
	Let us compute the atoms of $\mathcal B(C_2^3)$
	
	{\tiny
		\begin{lstlisting}[language=GAP, basicstyle=\ttfamily\tiny]
		gap> AtomsOfBlockMonoid([[1,0,0],[0,1,0],[1,1,0],[0,0,1],[1,0,1],[0,1,1],[1,1,1]],[2,2,2]);
		[ [ 0, 0, 0, 0, 0, 0, 2 ], [ 0, 0, 0, 0, 2, 0, 0 ], [ 0, 0, 0, 0, 0, 2, 0 ],
		[ 0, 0, 0, 2, 0, 0, 0 ], [ 0, 0, 2, 0, 0, 0, 0 ], [ 0, 2, 0, 0, 0, 0, 0 ],
		[ 2, 0, 0, 0, 0, 0, 0 ], [ 0, 0, 1, 0, 1, 1, 0 ], [ 0, 0, 1, 1, 0, 0, 1 ],
		[ 0, 1, 0, 0, 1, 0, 1 ], [ 0, 1, 0, 1, 0, 1, 0 ], [ 1, 0, 0, 0, 0, 1, 1 ],
		[ 1, 0, 0, 1, 1, 0, 0 ], [ 1, 1, 1, 0, 0, 0, 0 ], [ 0, 0, 0, 1, 1, 1, 1 ],
		[ 0, 1, 1, 0, 0, 1, 1 ], [ 0, 1, 1, 1, 1, 0, 0 ], [ 1, 0, 1, 0, 1, 0, 1 ],
		[ 1, 0, 1, 1, 0, 1, 0 ], [ 1, 1, 0, 0, 1, 1, 0 ], [ 1, 1, 0, 1, 0, 0, 1 ] ]
		\end{lstlisting}
	}
	
	So, from this point on, we ``live" inside $\mathbb N^7$ 
\end{frame}

